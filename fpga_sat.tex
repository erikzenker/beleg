\section{FPGA-SAT}
Es hat sich gezeigt, dass sich mithilfe von FPGAs und angepassten 
Algorithmen Probleme wesentlich effizienter lösen lassen
als mit Universal-Prozessoren \cite{preusser:2009}\cite{dietrich:2009}. 
So kann auch das SAT-Problem und dessen\\Lösungsalgorithmen 
auf die Verarbeitung in FPGAs angepasst werden.
In diesem Kapitel wird auf die bisherige Arbeit im Bereich
Hardware-SAT-Solver eingegangen und zwei Entwürfe, welche
am vielsversprechendsten sind, kurz erläutert. 
Eine Zusammenfassung der genutzten 
Techniken bis 2004 wird gegeben von
Skliarova \cite{ferrari:2004} und
bietet einen Überblick zum Thema FPGA-SAT.

\subsection{Kategorisierung von Hardware-SAT-Solvern}
Skilarova \cite{ferrari:2004} unterschiedet, auf welche Art
und Weise man SAT-Probleme mithilfe von Hardware löst. 
Diese Kategorisierung von Hardware-SAT-Solvern wird auch in dieser Arbeit
aufgegriffen. Unterschieden werden typische SAT-Algorithmen wie DPLL, CDCL bzw. SLS, 
aber auch verwendete Entscheidungsheuristiken bzw. andere Techniken moderner
Software-SAT-Solver.\\
Desweiteren unterscheidet man das genutzte Programiermodell.
Für jede SAT-Instanz kann ein neuer Schaltkreis 
generiert werden, dann spricht man vom instanzspezifischen Programmiermodell 
oder man entwirft einen Schaltkreis, welcher 
synthetisiert verschiedene Instanzen 
lösen kann, dann spricht man vom aplikationsspezifischen Modell.\\
Das dritte Unterscheidungsmerkmal ist das Ausführungsmodell. 
Man kann den kompletten SAT-Solver in Hardware auf einem FPGA 
realisieren,dann ist es eine reine Hardware Lösung, oder man lagert
Teilkomponenten in Hardware aus, 
dann wird von einer Hardware / Software-Hybrid-Lösung gesprochen.\\

\subsection{Vorhandene Hardware-SAT-Solver}
Erste reine Hardware-Lösungen im Zeitraum von 1996 bis 2004
waren meist instanzspezifische Solver und für Probleme bis 
100 Variablen ausgelegt. Instanzspezifische Entwürfe benötigen 
bei neuen Instanzen Zeit um eine Synthese durchzuführen, wobei 
das eigentliche Lösen des Problems dann sehr schnell geschieht.
Diese Hardware-SAT-Solver wurden entworfen bevor
moderne SAT-Solver \cite{manthey:2010} ihren Siegeszug in der
Software-Welt antraten, sodass es eine Zeit lang keine weiteren Veröffentlichungen 
gegeben hat und ein Performancevergleich nicht möglich war.\\
Ab 2008 findet man erste Arbeiten, welche bei der Problemkomplexität 
mit modernen Software-Solver mithalten können. 
Eine Arbeit von John D. Davis et al. \cite{davis:2008} 
beschreibt einen applikationspeziafischen Hybrid-Entwurf auf CDCL-Basis.
Ein Host-PC übernimmt die Berechnung von Entscheidungsvariablen und die Analyse 
von Konflikten. Auf den FPGA wurde die Inferenz-Komponente ausgelagert. 
Davis Solver kann Probleme mit bis zu 64\,k Variablen und 64\,k Klauseln lösen.\\
Eine weitere Arbeit von Leopold Haller et al. \cite{haller:2010} 
beschreibt auch einen applikationsspezifischen Entwurf, jedoch wurde 
kein Host-PC für verschiedene Berechnungen benutzt. Dieser   
Entwurf ist in reiner Hardware realisiert. Durch die Nutzung von großen 
DRAM-Speichern können Probleme von bis zu 1\,M Variablen und 70\,M 
Klauseln berechnet werden. Leider fehlen in dieser Arbeit Resultate.\\
Nach ausführlichem Literaturstudium stellt man fest,
dass es zwar FPGA-SAT-Implementierungen gibt und diese
zu einer Beschleunigung der Algorithmen führen, jedoch fehlt
der Vergleich zu modernen Software-SAT-Solvern.\\
%\todo{man könnte hier noch ausführlicher schreiben, zumindest bis
%die Seite voll ist}
