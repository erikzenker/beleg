\section{Einleitung}

In der Informatik beschäftigt man sich mit einer Vielzahl von
kombinatorischen Problemen und wie diese effizient gelöst werden können.
Nicht jedes Problem ist gleich schwer bzw. leicht zu lösen, weshalb sie in
sogenannte Problemklassen eingeteilt werden.  Eine Klasse ist die Klasse der
nichtdeterministisch polynomiell vollständigen Probleme. Für NP-vollständige Probleme gibt es bisher
keine Möglichkeit, diese effizient zu lösen. Ein Durchprobieren von
allen Lösungsmöglichkeiten würde sehr viel Zeit in Anspruch nehmen und
wird deshalb nur bei kleiner Problemgröße durchgeführt. Deshalb
werden oft problemspezifische Heuristiken benutzt, wodurch trotz
exponentieller Komplexität durchaus gute Ergebnisse erzielt werden
können.  Auch das Erfüllbarkeitsproblem (engl. Satisfiability Testing SAT)
gehört zur Klasse der NP-vollständigen Probleme. Moderne SAT Solver
können SAT-Probleme mit mehreren millionen Variablen und Klauseln
lösen und werden beständig weiter entwickelt.\\ Die
Rechnerarchitekturen auf denen SAT-Solver ausgeführt werden ändern
sich unaufhaltsam in Richtung Many-Core-Architekturen. Bereits heutige
Serversysteme haben 12 unabhängige Rechenkerne \cite{amdopteron:2011} auf einem
Prozessorchip, und selbst einfache Desktopsysteme sind standardmäßig
mit 2 bis 4 Kernen ausgestattet.\\ Ein Nachteil von SAT-Solvern ist
bisher, dass ihr Algorithmus auf sequentielle Verarbeitung
optimiert ist, obwohl bereits mehrere parallele Einheiten zur Verfügung
stehen.  Deshalb müssen die Algorithmen angepasst werden, um
auch in Zukunft die vorhandenen Ressourcen effektiv ausnutzen zu können.\\ Man
kann noch einen Schritt weiter gehen. Statt der Auslagerung der Algorithmen
auf einige wenige Kerne, nutzt man die hohe Parallelität
von Field Programmable Gate Arrays (FPGAs).  Ein Ansatz ist es den
Inferenzmechanismus, welcher ca. 80 bis 90\% der Rechenleistung in
aktuellen SAT-Solver belegt, auf einen FPGA auszulagern und 
ihn hochparallel auszuführen.\\
%\todo{Hier noch ein bisschen mehr zu FPGAs einleiten}\\

